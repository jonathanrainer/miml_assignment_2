\documentclass[11pt]{scrreprt}
\usepackage[margin = 1in]{geometry}              
\geometry{a4paper}                  
\usepackage[parfill]{parskip}    
\usepackage{fancyhdr}
\usepackage{enumitem}
\usepackage{longtable}
\usepackage{amsmath}
\usepackage{minted}
\usepackage{listings}
\usepackage{framed}
\usepackage{graphicx}
\usepackage[toc,page]{appendix}
\usepackage{lscape}
\usepackage{rotating}

\title{Assignment \#2 - Random Parameter Values, Random Matrices \& Monte-Carlo Integration}   
\date{\today}
\pagestyle{fancyplain}
\author{Jonathan Rainer}

\begin{document}
	\fancyhf{}
	\lhead{Jonathan Rainer}
	\chead{\thepage}
	\rhead{\today}
	\renewcommand{\headrulewidth}{0.3pt}
	\renewcommand{\footrulewidth}{0.3pt}
	\renewcommand{\footrulewidth}{0pt}
	\maketitle
	
	\begin{enumerate}
		\item Using the May-like random matrices developed in Week 5 Practical, and incorporating the sign structure used in the Aleesina and Tang interpretation of predator-prey and mutualistic interactions, verify numerically the key result of Allesina and Tang (2012). 
		
		[I.E. Show that under the assumtpions of Allesina and Tang (2012), predator-prey interactions are stabilising, and mutualistic interactions are destabilising.]
		
		
		
		\item Suppose than an unstructured random matrix of the form considered by Allesian and Tang (i.e. a matrix leading to the circular eigenvalue spectrum in Figure 1a) is modified by having the elements of each row multiplied by a factor $R_j$, where $R_j \in [0, 2]$ is a uniform random variable chosen independently for each row. Evaluate the consequences of this modification for stability (in the sense of Allesian and Tang), in comparison to the roles of mutualism and predator-prey interactions.
		
		\item Consider the Thebault and Fontaine model of obligate mutualism used in James et al. (2012), as detailed in the Supplementary Information. The authors claim that (SI, p12.): 
		
		\emph{``Moreover, simple simulations on systems with only a single plant species show that an isolated mutualisitic pair of species, each with no other partners, has an extinction probability of over 99\%'' }
		
		Solve the system numerically and check whether this claim is true.
		
		\item Explain briefly why the models considered by Coyte et.al (2015) (as detailed in Section Method 1a of the Supplementary Material, pages 4 and 5) are vulnerable to the criticisms about diagonal elements in the Jacobian as outlined in James et. al (2015). Do the main conclusion of Coyte et al. (2015) change qualitatively when diagonal variability is added? Explain your answer with the support of appropriate numerical outputs.
		
	\end{enumerate}
	
	
	\begin{appendices}
		\chapter{Programming Code}
		
	\end{appendices}
\end{document}
